% A poster using beamerposter and the gemini theme with epfl colors.
% Beamerposter will let you create a poster in a single beamer frame
% using the columns and blocks from the beamer class.
% The color theme is adapted from Gemini's MIT scheme.

% use beamer as documentclass
\documentclass[final,cmyk]{beamer}

% use the beamerposter package to create a poster
\usepackage[scale=1.4,orientation=portrait,size=a0]{beamerposter}
\usepackage{csquotes}
\usepackage[backend=biber,firstinits=true,maxbibnames=99,bibencoding=utf8,style=numeric]{biblatex}
\addbibresource{references.bib}
% make references small
\renewcommand*{\bibfont}{\scriptsize}

% use the gemini theme. 
\usetheme{gemini}
% use the epfl color theme, defining epfl{dark}red, -{dark}green,  -light, -gray, and -dark
\usecolortheme{epfl}

\usepackage{graphicx}
\usepackage{tikz}
\usepackage{adjustbox}
\usepackage{qrcode}
\usepackage{wrapfig}

\newcommand*{\pianoroll}{
  \draw (0,2) rectangle (1,2.4);
  \draw (1,2.4) rectangle (2,2.8);
  \draw (2,2) rectangle (2.5,2.4);
  \draw (2.5,1.6) rectangle (3,2);
  \draw (3,1.2) rectangle (3.5,1.6);
  \draw (3.5,0.8) rectangle (4,1.2);
  
  \draw (4,1.6) rectangle (5,2);
  \draw (5,2) rectangle (6,2.4);
  \draw (6,1.6) rectangle (6.5,2);
  \draw (6.5,1.2) rectangle (7,1.6);
  \draw (7,0.8) rectangle (7.5,1.2);
  \draw (7.5,0.4) rectangle (8,0.8);
  
  \draw (1,-0.4) rectangle (2,0);
  \draw (2,0) rectangle (4,0.4);
  \draw (5,-0.8) rectangle (6,-0.4);
  \draw (6,-0.4) rectangle (8,0);
}

\newcommand{\sgcross}[1]
{
  \begin{scope}[xshift=#1cm,thick,color=epflred]
    \draw (0.35,0.1) -- (0.55,0.3);
    \draw (0.55,0.1) -- (0.35,0.3);
  \end{scope}
}
\newcommand{\sgbox}[1]
{
  \begin{scope}[xshift=#1cm]
    \draw (0,0) rectangle (0.9,0.4);
  \end{scope}
}
\newcommand{\sgboxfill}[2]
{
  \begin{scope}[xshift=#2cm]
    \draw[fill=#1] (0,0) rectangle (0.9,0.4);
  \end{scope}
}
\newcommand{\sgrow}[4][epflgray]
{
  \begin{scope}[yshift=#2cm]
    \foreach \x in {0,...,8} { \sgbox{\x} }
    \foreach \x in {#3} { \sgboxfill{#1}{\x} }
    \foreach \x in {#4} { \sgcross{\x} }
  \end{scope}
}

\title{Generalized Skipgrams}

\author{Christoph Finkensiep, Markus Neuwirth, Martin Rohrmeier}

\institute{Digital and Cognitive Musicology Lab, École Polytechnique Fédérale de Lausanne}


\begin{document}

\begin{frame}[t]

  \begin{columns}[t]
    \begin{column}{0.3\textwidth}
      \begin{block}{Problem}
        Many aspects of music are oriented \alert{vertically}
        (harmony), \alert{horizontally} (melody), or both (voice
        leading).  However, the structure of music is neither cleanly
        vertical nor horizontal, but rather irregular in both
        dimensions.

        \begin{adjustbox}{scale=2.7,center,margin=0 1ex}
          \begin{tikzpicture}
            \begin{scope}[color=epflred,thick,->,>=stealth]
              \draw[<->] (1.5,2.2) -- (1.5,0.2);
              \draw (4.5,1.8) -- (5.5,2.2);
              \draw (5.5,2.2) -- (6.25,1.8);
              \draw (6.25,1.8) -- (6.75,1.4);
            \end{scope}
            
            \pianoroll
          \end{tikzpicture}
        \end{adjustbox}

        Vertical structure is usually enforced by \alert{slicing}.
        This can be problematic for cutting through notes and in cases
        where vertically related notes do not overlap (and thus do not
        have a common slice).

        \begin{adjustbox}{scale=2.7,center,margin=0 1ex}
          \begin{tikzpicture}
            \foreach \x in {0,1,2,2.5,3,3.5,4,5,6,6.5,7.5,8}
              \draw[color=epflred] (\x,-1) -- (\x,3);
            \pianoroll
          \end{tikzpicture}
        \end{adjustbox}

        Horizontal structure is often ensured by working on
        \alert{monophonic voices} or just a single \alert{melody}.
        This is impractical in the general case, e.g., in piano music.
        
        \begin{adjustbox}{scale=2.7,center,margin=0 1ex}
          \begin{tikzpicture}
            \pianoroll
            \draw[color=epflred,very thick] (0,0.4) -- (8,0.4);
          \end{tikzpicture}
        \end{adjustbox}
        
      \end{block}

      \vspace{5em}

      \begin{block}{Funding and Paper Link}
        \begin{wrapfigure}{l}{0.3\textwidth}
          \begin{center}
            \qrcode[height=0.25\textwidth]{http://ismir2018.ircam.fr/doc/pdfs/202_Paper.pdf}
          \end{center}
        \end{wrapfigure}
        \small
        The research presented on this poster is generously
        supported by the Volkswagen Foundation.  We also thank
        Claude Latour for supporting this research through the
        Latour Chair in Digital Musicology.  The paper to this
        poster including all references can be found using the QR
        code to the left.
        
        
        \includegraphics[width=\textwidth]{../logos/vw/vw.pdf}
        
        \includegraphics[width=\textwidth]{../logos/epfl/logo_print.pdf}
      \end{block}
    \end{column}
    
    \begin{column}{0.3\textwidth}
      \begin{block}{Solution}
        We generalize the \alert{skipgram technique},
        well known in linguistics \autocite{GuthrieCloserLookSkipGram2006}
        and recently introduced to music research \autocite{SearsModelingHarmonySkipGrams2017}.
        Skipgrams are similar to $n$-grams (fixed length subsequences of a sequence)
        but allow a limited amount of \alert{gaps} between their elements.

        \begin{adjustbox}{scale=2.7,center,margin=0 1ex}
          \begin{tikzpicture}
            \sgrow{-3.5}{1,3,7}{2,4,5,6}
    
            \begin{scope}[yshift=-2cm]
              \sgboxfill{epflgray}{2.5}
              \sgboxfill{epflgray}{3.5}
              \sgboxfill{epflgray}{4.5}
            \end{scope}
            \begin{scope}[->,>=stealth]
              \draw[shorten >=0.5cm,shorten <=0.5cm] (1.45,-3.3) -- (2.95,-1.8);
              \draw[shorten >=0.4cm,shorten <=0.4cm] (3.45,-3.3) -- (3.95,-1.8);
              \draw[shorten >=0.8cm,shorten <=0.8cm] (7.45,-3.3) -- (4.95,-1.8);
            \end{scope}
          \end{tikzpicture}
        \end{adjustbox}

        If instead of counting gaps the \enquote{amount of skip} is defined by a
        \alert{skip function}, the technique can be applied to non-sequential structures.
        The skip function is applied to consecutive elements in the skipgram.
        The summed skip must not exceed some parameter $k$.

        \begin{adjustbox}{scale=2.7,center,margin=0 0 0 1ex}
          \begin{tikzpicture}
            \pianoroll

            \draw[fill=epflgray] (0,2) rectangle (1,2.4);
            \draw[fill=epflgray] (6,1.6) rectangle (6.5,2);
            \draw[fill=epflgray] (2,0) rectangle (4,0.4);
            
            \draw (0,2) -- (0,0.4);
            \draw (2,0.4) -- (2,0.8);
            \draw (6,1.6) -- (6,0.4);
            \draw[<->,color=epflred] (0,0.6) -- (2,0.6);
            \draw[<->,color=epflred] (2,0.6) -- (6,0.6);
          \end{tikzpicture}
        \end{adjustbox}
        \begin{adjustbox}{scale=1.45,margin=0 0 0 1ex}
          \begin{tikzpicture}[overlay]
            \node () at (3,4.2) {\scriptsize $d_1$};
            \node () at (10,4.2) {\scriptsize $d_2$};
            \node () at (7.5,1.2) {\scriptsize $d_1 + d_2 \leq k$};
          \end{tikzpicture}
        \end{adjustbox}
        
        Vertical structure can now be described by \alert{skipgrams over notes} in a piece.
        An appropriate skip function could be the difference between the notes' onsets.
        This generates fixed-size groups of notes with \alert{limited} (but possible)
        \alert{non-simultaneity}.

        \begin{adjustbox}{scale=2.7,center,margin=0 1ex}
          \begin{tikzpicture}
            \draw[thick,color=epflred] (0.5,2.2) -- (1.5,-0.2);
            \pianoroll
            \draw[fill=epflgray] (0,2) rectangle (1,2.4);
            \draw[fill=epflgray] (1,-0.4) rectangle (2,0);
          \end{tikzpicture}
        \end{adjustbox}
        
        Horizontal structure can be extracted much like vertical structure.
        Additionally requiring the notes not to overlap enforces \alert{sequentiality}.

        \begin{adjustbox}{scale=2.7,center,margin=0 1ex}
          \begin{tikzpicture}
            \draw[thick,color=epflred] (0.5,2.2) -- (2.75,1.8);
            \pianoroll
            \draw[fill=epflgray] (0,2) rectangle (1,2.4);
            \draw[fill=epflgray] (2.5,1.6) rectangle (3,2);
          \end{tikzpicture}
        \end{adjustbox}
        
        Horizontal and vertical structure can be combined
        by \alert{recursive application of skipgrams}.
        A first pass generates vertical structure as skipgrams over notes (\enquote{stages}).
        A second step generates horizontal structure as skipgrams over stages,
        resulting in a nested, two-dimensional pattern.

        \begin{adjustbox}{scale=2.7,center,margin=0 1ex}
          \begin{tikzpicture}
            \begin{scope}[thick]
              \draw (0.5,2.2) -- (1.5,-0.2);
              \draw (2.75,1.8) -- (3,0.2);
              \draw (4.5,1.8) -- (5.5,-0.6);
              \draw[->,color=epflred,>=stealth] (1.25,1) -- (2.5,1);
              \draw[->,color=epflred,>=stealth] (3.25,1) -- (4.5,1);
            \end{scope}
            \pianoroll
            \draw[fill=epflgray] (1,-0.4) rectangle (2,0);
            \draw[fill=epflgray] (2,0) rectangle (4,0.4);
            \draw[fill=epflgray] (0,2) rectangle (1,2.4);
            \draw[fill=epflgray] (2.5,1.6) rectangle (3,2);
            \draw[fill=epflgray] (4,1.6) rectangle (5,2);
            \draw[fill=epflgray] (5,-0.8) rectangle (6,-0.4);
          \end{tikzpicture}
        \end{adjustbox}
      \end{block}
    \end{column}
    
    \begin{column}{0.3\textwidth}
      \begin{block}{Basic Algorithm}
        The algorithm for enumerating skipgrams takes as input:

        \begin{itemize}
        \item a list $L$ of objects to generate skipgrams over
        \item a skip function $f$ on pairs of objects
        \item the skipgram length $n$
        \item the skip limit $k$
        \end{itemize}

        The list $L$ must be ordered with respect to the skip function:
        \[\forall i < j < k: f(L_i,L_j) \leq f(L_i,L_z).\]
        
        The algorithm returns all sublists $x$ of $L$ with $|x| = n$ and
        $\sum_i f(x_i,x_{i+1}) \leq k$.

        $L$ is traversed, building up a \alert{list of prefixes}
        that eventually become complete skipgrams.

        For each element $e$ in $L$:
        \begin{enumerate}
        \item remove old prefixes that cannot be extended with $e$
        \item extend all remaining prefixes with $e$%; of those:
          %\begin{enumerate}
          \item output all completed prefixes (length $n$)
          \item add all incomplete prefixes to the prefix list
          %\end{enumerate}
        \item add a new prefix $[e]$.
        \end{enumerate}

        \begin{adjustbox}{scale=2,center,margin=0 1ex}
          \begin{tikzpicture}
            \begin{scope}[scale=0.3]
              \pianoroll
              \draw[fill=epflgray] (1,-0.4) rectangle (2,0);
              \draw[fill=epflgray] (2,2) rectangle (2.5,2.4);
            \end{scope}
            
            \begin{scope}[scale=0.3,yshift=4cm]
              \fill[epfllight] (-0.25,-1) rectangle (8.25,3);
              \pianoroll
              \draw[fill=epflred] (0,2) rectangle (1,2.4);
              \draw[fill=epflred] (2,2) rectangle (2.5,2.4);
            \end{scope}

            \begin{scope}[scale=0.3,yshift=8cm]
              \pianoroll
              \draw[fill=epflgray] (1,-0.4) rectangle (2,0);
              \draw[fill=epflgray] (1,2.4) rectangle (2,2.8);
            \end{scope}

            \draw (2.75,-0.4) -- (2.75,3.2);

            \begin{scope}[xshift=3cm,yshift=0.4cm]
              \pianoroll
              \draw[fill=epflgray] (0,2) rectangle (1,2.4);
              \draw[fill=epflgray] (2,2) rectangle (2.5,2.4);
              \draw[fill=epflred] (3,1.2) rectangle (3.5,1.6);
            \end{scope}
          \end{tikzpicture}
        \end{adjustbox}
      \end{block}

      \begin{block}{Extensions}
        Efficient \alert{filtering} can be implemented
        by \alert{testing a predicate} on every extension of a prefix.
        If the new prefix does not satisfy the predicate, it is discarded.

        \alert{Sampling} can be implemented efficiently
        by \alert{flipping a coin} on every prefix extension,
        deciding whether to keep or to discard the prefix.
        A prefix is extended $n-1$ times,
        so keeping each prefix with probability $\sqrt[n-1]{p}$
        means keeping the skipgram with probability $p$.

        The output order depends on the last element of each skipgram,
        because the algorithm outputs skipgrams when they are completed.
        If the \alert{order of the initial elements} should be retained,
        completed skipgrams are first entered in a \alert{priority queue}.
        In each iteration, only those skipgrams are taken from the queue
        that cannot be preceded by currently active prefixes anymore.
      \end{block}

      \begin{block}{References}
          \printbibliography
      \end{block}
    \end{column}
  \end{columns}
  
\end{frame} % End of the enclosing frame

\end{document}